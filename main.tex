\documentclass{fefu}
\author{Терехов Д.Е.}
\setschool{ШКОЛА ЕСТЕСТВЕННЫХ НАУК ДВФУ}
\setdepartment{кафедра информатики, математического и компьютерного моделирования}{Чеботарев}
\setgroup{Б8403а}

\begin{document}
\makereporttitle
\tableofcontents
\pagebreak
\section*{Аннотация}
В компьютерной графике объекты состоят из полигонов, чаще всего треугольников.
На видеокарту загружается текстурный атлас -- совокупность текстур меньшего размера.
Для уменьшения количества текстурных атласов и вызовов на отрисовку необходимо как можно плотнее упаковать
текстуры в атласы. Для этого предварительно текстуры необходимо триангулировать текстуры -- таким образом
 она будет занимать меньшее пространство, а также фрагментному шейдеру необходимо будет отрисовать меньше пикселей.
\end{document}
