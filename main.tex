\documentclass{fefu}
\author{Терехов Д.Е.}
\setschool{ШКОЛА ЕСТЕСТВЕННЫХ НАУК ДВФУ}
\setdepartment{кафедра информатики, математического и компьютерного моделирования}{Чеботарев}
\setgroup{Б8403а}

\begin{document}
\makereporttitle
\tableofcontents
\pagebreak
\section*{Аннотация}
В компьютерной графике объекты состоят из полигонов, чаще всего треугольников.
На видеокарту загружается текстурный атлас -- совокупность текстур меньшего размера.
Для уменьшения количества текстурных атласов и вызовов на отрисовку необходимо как можно плотнее упаковать
текстуры в атласы. Для этого предварительно текстуры необходимо триангулировать текстуры -- таким образом
 она будет занимать меньшее пространство, а также фрагментному шейдеру необходимо будет отрисовать меньше пикселей.
\section{Введение}
\subsection{Глоссарий}
\subsection{Описание предметной области}
\subsubsection{Студия "Game Forest"}
Работа выполняется по заказу студии "Game Forest". TODO
\subsubsection{Citrus Game Engine}
Citrus -- игровой движок с открытым исходным кодом, распространяемый по лицензии GPL-3.0.
репозиторий размещен на платформе Github \cite{CitrusRepo}. Citrus Game Engine
состоит из следующих компонент:
\begin{itemize}
    \item Lime -- ядро игрового движка
    \item Lemon -- линкер сторонних библиотек
    \item Yuzu -- библиотека для сериализации
    \item Orange -- сборщик приложений
    \item Tangerine -- редактор сцен
    \item Kumquat -- генератор кода
\end{itemize}
\subsubsection{Полигонизация текстур}
\newpage
\bibliographystyle{ugost2008ls}
\bibliography{references}
\end{document}
